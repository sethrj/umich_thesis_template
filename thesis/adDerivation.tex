% !TEX root = _individual/adDerivation.tex

%%%%%%%%%%%%%%%%%%%%%%%%%%%%%%%%%%%%%%%%%%%%%%%%%%%%%%%%%%%%%%%%%%%%%%%%%%%%%%%%
\chapter{Anisotropic diffusion}\label{chap:adDerivation}

Chapter text goes here.

An example of a figure using proper, typographically appealing rules with the
\verb|booktabs| package is Table~\ref{tab:angularDomain}.

\begin{table}[htb]
  \centering
  \begin{tabular}{rccc} \toprule
   Geometry & $\vec{\Omega}$ & Domain $S$ & $\ud\Omega$
\\ \midrule
   1-D & $\mu$ & $-1 \le \mu \le 1$ & $\ud\mu$
   \\
   2-D & $\sqrt{1-\mu^2} \cos \omega \vec{i}
   + \sqrt{1-\mu^2} \sin \omega \vec{j}$
   & $-1 \le \mu \le 1$, $0 \le \omega < 2\pi$ & $\ud\mu \ud \omega$
   \\
   Flatland & $\cos \omega \vec{i} + \sin \omega \vec{j}$
   & $0 \le \omega < 2\pi$ & $\ud \omega$
   \\
   3-D & $\mu \vec{i}
   + \sqrt{1-\mu^2} \cos \omega \vec{j}
   + \sqrt{1-\mu^2} \sin \omega \vec{k}$
   & $-1 \le \mu \le 1$, $0 \le \omega < 2\pi$ & $\ud\mu \ud \omega$
\\ \bottomrule
  \end{tabular}
  \caption{Angular variables in the various geometries.}
  \label{tab:angularDomain}
\end{table}


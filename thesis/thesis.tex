%\documentclass[twoside]{umthesis}
\documentclass{umthesis}
%\documentclass[draft]{umthesis}
%%%%%%%%%%%%%%%%%%%%%%%%%%%%%%%%%%%%%%%%%%%%%%%%%%%%%%%%%%%%%%%%%%%%%%%%%%%%%%%%
\usepackage[inner=1.5in,outer=1in,top=1in,bottom=1in,headheight=0pt]{geometry}

\usepackage{amsmath}
% font choice
\usepackage[T1]{fontenc}
%\usepackage{txfonts}
\usepackage{fourier}
\SetMathAlphabet{\mathcal}{normal}{OMS}{cmsy}{m}{n}%revert calligraphy math

% \singlespacing
\doublespacing
%%%%%%%%%%%%%%%%%%%%%%%%%%%%%%%%%%%%%%%%%%%%%%%%%%%%%%%%%%%%%%%%%%%%%%%%%%%%%%%%
% extra packages and shortcuts
\usepackage{bm}
\usepackage{amssymb}
\usepackage{microtype}
\usepackage[pdftex]{graphicx}
\usepackage[width=.75\textwidth,font=small,labelfont=bf]{caption}
\usepackage{booktabs} % \toprule, \midrule, \bottomrule
\usepackage{verbatim}
% note: rotating must go AFTER graphicx
\usepackage{subfig} % figure 1a, 1b
\usepackage{rotating} % sideways tables
% note: this file is in $(git rev-parse --show-toplevel)/texmf/etc
\input{SRJinclude.tex}
%\setSRJthesisfigurepaths

% For use in the introduction
\newcommand{\chaptersynopsis}[1]{%
\subsubsection{Chapter~\ref{#1}: \nameref{#1}}}
% For referencing equations from a previous chapter; defined like this for
% compatibility with ``single-chapter'' compile
\newcommand{\tagref}[1]{\tag{\ref{#1}}}

% float setup
\renewcommand\floatpagefraction{.70}
\renewcommand\topfraction{.95}
\renewcommand\bottomfraction{.95}
\renewcommand\textfraction{.1}

% Improve list and enum spacing
\renewcommand{\itemhook}{\par\setstretch{1}%
  \setlength{\topsep}{0pt}%
  \setlength{\parskip}{0pt}%
  \setlength{\partopsep}{.5\baselineskip}%
  \setlength{\parsep}{0pt}%
  \setlength{\itemsep}{0pt}%
}
\renewcommand{\enumhook}{\itemhook}

\makeatletter
\renewcommand\section{\@startsection {section}{1}{\z@}%
                                     {-2.5ex  \@minus -.2ex}%
                                     {1.5ex \@minus .2ex}%
                                     {\normalfont\Large\bfseries}}
\renewcommand\subsection{\@startsection{subsection}{2}{\z@}%
                                     {-1.75ex  \@minus -.2ex}%
                                     {1.25ex \@minus .2ex}%
                                     {\normalfont\large\bfseries}}
\renewcommand\subsubsection{\@startsection{subsubsection}{3}{\z@}%
                                     {-1.5ex  \@minus -.2ex}%
                                     {1.25ex \@minus .2ex}%
                                     {\normalfont\normalsize\bfseries}}
\makeatother  
%%%%%%%%%%%%%%%%%%%%%%%%%%%%%%%%%%%%%%%%%%%%%%%%%%%%%%%%%%%%%%%%%%%%%%%%%%%%%%%%
\author{Seth R.~Johnson}
\title{Anisotropic Diffusion Approximations for Time-dependent Particle
Transport\texorpdfstring{\\%
  \emph{DRAFT: \today}}{}
}

\program{Nuclear Engineering and Radiological Sciences}
\degree{Doctor of Philosophy}
\chaircommitteemember{Edward W.~Larsen}{Professor}
\committeemember{Thomas J.~Downar}{Professor}
\committeemember{James P.~Holloway}{Professor}
\committeemember{William R.~Martin}{Professor}
\committeemember{Katsuyo S.~Thornton}{Professor}

%%%%%%%%%%%%%%%%%%%%%%%%%%%%%%%%%%%%%%%%%%%%%%%%%%%%%%%%%%%%%%%%%%%%%%%%%%%%%%%%
\includeonly{%
introduction,%
adDerivation,%
conclusion,%
}
%%%%%%%%%%%%%%%%%%%%%%%%%%%%%%%%%%%%%%%%%%%%%%%%%%%%%%%%%%%%%%%%%%%%%%%%%%%%%%%%
\begin{document}

% let ``align'' break anywhere
\allowdisplaybreaks
% keep huge spaces out from between paragraphs, especially when using double
% spacing
\setlength{\parskip}{0pt plus 0pt minus 0pt}

%%%%%%%%%%%%%%%%%%%%%%%%%%%%%%%%%%%%%%%%%%%%%%%%%%%%%%%%%%%%%%%%%%%%%
% FRONT MATTER
\frontmatter

\maketitle

%%%%%%%%%%%%%%%%%%%%%%%%%%%%%

\begin{frontispiece}
\begin{flushleft}
For in much wisdom is much vexation,\hfill\\
\hspace{1.5em}and he who increases knowledge increases sorrow.
\end{flushleft}
---Eccl.~1:18
%THIS PAGE INTENTIONALLY NOT LEFT BLANK
\end{frontispiece}

%%%%%%%%%%%%%%%%%%%%%%%%%%%%%

\begin{dedication}
  To my parents, Ayn Rand and God.
\end{dedication}

%%%%%%%%%%%%%%%%%%%%%%%%%%%%%

\begin{acknowledgments}
  Acknowledge people here.
\end{acknowledgments}

%%%%%%%%%%%%%%%%%%%%%%%%%%%%%

%\begin{preface}
%  before reading this, you should know\dots
%\end{preface}

% list of contents, etc
\tableofcontents
\listoftables
\listoffigures
%\listofappendices

% % the optional normal abstract is formatted the same as preface and acknowledgements,
% % and is listed in the table of contents
% \begin{abstract}
% \end{abstract}

%%%%%%%%%%%%%%%%%%%%%%%%%%%%%%%%%%%%%%%%%%%%%%%%%%%%%%%%%%%%%%%%%%%%%
% MAIN MATTER
\mainmatter

% note that chapter markers MUST go inside the ``include''-d file
% !TEX root = _individual/introduction.tex

%%%%%%%%%%%%%%%%%%%%%%%%%%%%%%%%%%%%%%%%%%%%%%%%%%%%%%%%%%%%%%%%%%%%%%%%%%%%%%%%
\chapter{Introduction}\label{chap:introduction}
%%%%%%%%%%%%%%%%%%%%%%%%%%%%%%%%%%%%%%%%%%%%%%%%%%%%%%%%%%%%%%%%%%%%%%%%%%%%%%%%

Introduction has citations \cite{Abb1884}.

%%%%%%%%%%%%%%%%%%%%%%%%%%%%%%%%%%%%%%%%%%%%%%%%%%%%%%%%%%%%%%%%%%%%%%%%%%%%%%%%
\section{Synopsis}

The remainder of this thesis is organized into the following chapters.

\chaptersynopsis{chap:adDerivation}
With the transport equation in hand, we derive a new approximation to radiation
transport, anisotropic diffusion. The derivation accounts for both time
dependence and boundary conditions. We then discuss some of the properties of
the AD method and make predictions for its range of applicability.

\chaptersynopsis{chap:conclusion}
The final chapter summarizes the results of the theory developed in this thesis
and its application to TRT problems. We discuss possible improvements to the new
methods and other future work.


\include{trtBackground}
% !TEX root = _individual/adDerivation.tex

%%%%%%%%%%%%%%%%%%%%%%%%%%%%%%%%%%%%%%%%%%%%%%%%%%%%%%%%%%%%%%%%%%%%%%%%%%%%%%%%
\chapter{Anisotropic diffusion}\label{chap:adDerivation}

Chapter text goes here.

An example of a figure using proper, typographically appealing rules with the
\verb|booktabs| package is Table~\ref{tab:angularDomain}.

\begin{table}[htb]
  \centering
  \begin{tabular}{rccc} \toprule
   Geometry & $\vec{\Omega}$ & Domain $S$ & $\ud\Omega$
\\ \midrule
   1-D & $\mu$ & $-1 \le \mu \le 1$ & $\ud\mu$
   \\
   2-D & $\sqrt{1-\mu^2} \cos \omega \vec{i}
   + \sqrt{1-\mu^2} \sin \omega \vec{j}$
   & $-1 \le \mu \le 1$, $0 \le \omega < 2\pi$ & $\ud\mu \ud \omega$
   \\
   Flatland & $\cos \omega \vec{i} + \sin \omega \vec{j}$
   & $0 \le \omega < 2\pi$ & $\ud \omega$
   \\
   3-D & $\mu \vec{i}
   + \sqrt{1-\mu^2} \cos \omega \vec{j}
   + \sqrt{1-\mu^2} \sin \omega \vec{k}$
   & $-1 \le \mu \le 1$, $0 \le \omega < 2\pi$ & $\ud\mu \ud \omega$
\\ \bottomrule
  \end{tabular}
  \caption{Angular variables in the various geometries.}
  \label{tab:angularDomain}
\end{table}


\include{aponeDerivation}
\include{implementation}
\include{flatland}
\include{simpleNumericalResults}
\include{trtNumericalResults}
% !TEX root = _individual/conclusion.tex

%%%%%%%%%%%%%%%%%%%%%%%%%%%%%%%%%%%%%%%%%%%%%%%%%%%%%%%%%%%%%%%%%%%%%%%%%%%%%%%%
\chapter{Conclusions}\label{chap:conclusion}
%%%%%%%%%%%%%%%%%%%%%%%%%%%%%%%%%%%%%%%%%%%%%%%%%%%%%%%%%%%%%%%%%%%%%%%%%%%%%%%%

Our new method is simply the best ever! Please let me graduate.



%\appendix
%\include{pytrt}

%%%%%%%%%%%%%%%%%%%%%%%%%%%%%%%%%%%%%%%%%%%%%%%%%%%%%%%%%%%%%%%%%%%%%%%%%%%%%%%%
% BIBLIOGRAPHY
\backmatter
\bibliographystyle{ans}
\bibliography{../SRJall}
%%%%%%%%%%%%%%%%%%%%%%%%%%%%%%%%%%%%%%%%%%%%%%%%%%%%%%%%%%%%%%%%%%%%%%%%%%%%%%%%
% OPTIONAL INDEX
%\phantomsection %makes sure it points to the right page
%\addtocontents{toc}{chapter}{Index}
%\printindex
%%%%%%%%%%%%%%%%%%%%%%%%%%%%%%%%%%%%%%%%%%%%%%%%%%%%%%%%%%%%%%%%%%%%%%%%%%%%%%%%
\end{document}

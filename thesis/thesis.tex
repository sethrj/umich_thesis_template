%\documentclass[twoside]{umthesis}
\documentclass{umthesis}
%\documentclass[draft]{umthesis}
%%%%%%%%%%%%%%%%%%%%%%%%%%%%%%%%%%%%%%%%%%%%%%%%%%%%%%%%%%%%%%%%%%%%%%%%%%%%%%%%
\usepackage[inner=1.5in,outer=1in,top=1in,bottom=1in,headheight=0pt]{geometry}

\usepackage{amsmath}
% font choice
\usepackage[T1]{fontenc}
%\usepackage{txfonts}
\usepackage{fourier}
\SetMathAlphabet{\mathcal}{normal}{OMS}{cmsy}{m}{n}%revert calligraphy math

% \singlespacing
\doublespacing
%%%%%%%%%%%%%%%%%%%%%%%%%%%%%%%%%%%%%%%%%%%%%%%%%%%%%%%%%%%%%%%%%%%%%%%%%%%%%%%%
% extra packages and shortcuts
\usepackage{bm}
\usepackage{amssymb}
\usepackage{microtype}
\usepackage[pdftex]{graphicx}
\usepackage[width=.75\textwidth,font=small,labelfont=bf]{caption}
\usepackage{booktabs} % \toprule, \midrule, \bottomrule
\usepackage{verbatim}
% note: rotating must go AFTER graphicx
\usepackage{subfig} % figure 1a, 1b
\usepackage{rotating} % sideways tables
% note: this file is in $(git rev-parse --show-toplevel)/texmf/etc
%%% INCLUDE FILE FOR DEFINITIONS
%%% These may require various packages.

% Shortcuts in regular text
\newcommand{\degs}{\ensuremath{^\circ}}
\newcommand{\EE}[1]{\ensuremath{\times 10^{#1}}}
\newcommand{\ttimes}{\ensuremath{{}\times{}}}
\newcommand{\cclicense}{%
  \smash{\raisebox{-0.45ex}{%
  \setlength{\unitlength}{1em}%
  \begin{picture}(1,1)%
    \put(0.5,0.5){\circle{1}}
    \put(0.5,0.5){\hbox to 0pt{\hss\raisebox{-.45ex}{\tiny\textsf{CC}}\hss}}
  \end{picture}%
  }}%
  \hskip -1em%
  \href{http://creativecommons.org/licenses/by-nc-sa/3.0/}%
  {\ \hskip 1em \textsf{BY-NC-SA}}%
}

%\newcommand{\horizsep}{{\par\noindent\centering\rule[.25ex]{.75\columnwidth}{2pt}\par}}
\newcommand{\horizsep}{\vspace{\baselineskip}\noindent\hspace{\stretch{1}}$
\ast\qquad \ast\qquad \ast\qquad
$ \hspace{\stretch{1}} \vspace{\baselineskip}}
\newcommand{\pytrt}{\textsf{PyTRT}}

% Research
%\newcommand{\lop}[1]{\mathcal{L}\!\left[#1\right]}
\newcommand{\lopinv}[2]{\mathcal{I}_{#1}\!\left[#2\right]}
\newcommand{\Dtens}{\mat{D}}
\newcommand{\Etens}{\mat{E}}
\newcommand{\Identitytens}{\mat{I}}
\newcommand{\APone}{AP$_1$}
\newcommand{\Pone}{P$_1$}
\newcommand{\SN}{S$_N$}%{S$_\text{N}$}%{$S_N$}%
\newcommand{\PN}{P$_N$}%{P$_\text{N}$}%{$P_N$}%
%\newcommand{\CN}{Crank--Nicolson} %Yes, it's Nic not Nich
%\newcommand{\Eddington}{E} %whatever symbol I decided for Eddington
\newcommand{\raden}{\mathcal{E}} %radiation energy
%\newcommand{\Sigmatr}{\Sigma_{\mathit{tr}}}
\newcommand{\sigmast}{\sigma^{*\!}}
\newcommand{\xy}{$x$--$y$}
\newcommand{\xyz}{$x$--$y$--$z$}
\newcommand{\rz}{$r$--$z$}

% Set include file paths for thesis
% graphics paths
% \makeatletter
% \newcommand{\setSRJthesisfigurepaths}{%
% \graphicspath{{/Users/seth/_thesis/figures/}}%
% \def\input@path{{/Users/seth/_thesis/figures/}}%
% }
\newcommand{\Iv}{I_\mathrm{v}}
\newcommand{\Ibl}{I_\mathrm{bl}}
\newcommand{\Iil}{I_\mathrm{il}}


% Program names
\newcommand{\cpp}{\textsf{C\raisebox{0.2ex}{++}}}

% General math shortcuts
\newcommand{\ud}{\mathop{}\!\mathrm{d}}
\newcommand{\pder}[2]{\frac{\partial #1}{\partial #2}}
\newcommand{\oder}[2]{\frac{\mathrm{d} #1}{\mathrm{d} #2}}
\newcommand{\tpder}[2]{{\partial #1}/{\partial #2}} %inlined
\newcommand{\toder}[2]{{\mathrm{d} #1}/{\mathrm{d} #2}} %inlined
\newcommand{\lra}{ \quad \Longrightarrow \quad }
\newcommand{\eexp}{\mathop{}\!\mathrm{e}} % upright ``e'' for exponent
\newcommand{\expp}[1]{\exp\!\left( {#1} \right)} % exp with parentheses
\newcommand{\qeq}{\stackrel{\mathrm{?}}{=}}

% Probability
%\newcommand{\expectation}[1]{\mathop{}\!\mathrm{E}\!\left[ #1 \right]}
%\DeclareMathOperator{\Var}{Var} % variance

% Asymptotic analysis
%\DeclareMathOperator{\Ei}{Ei} % Exponential function
%\newcommand{\lapl}[1]{\mathcal{L}[{#1}]} %laplace

%change the Re and Im operators from fancy curly letters
\DeclareMathOperator{\MathOpRe}{Re}
\renewcommand{\Re}{\MathOpRe}
\DeclareMathOperator{\MathOpIm}{Im}
\renewcommand{\Im}{\MathOpIm}

%imaginary ``i'' , upright 'i' or \imath
\newcommand{\iimag}{\mathrm{i}}

% Finite differences
%\newcommand{\hot}{\text{h.o.t.}}
\newcommand{\inv}{^{-1}}

% Numerical Linear Algebra
\newcommand{\conj}{^{\ast}} % complex conjugate (transpose)
\newcommand{\norm}[1]{\left\| #1 \right\|} % double pipe
\newcommand{\abs}[1]{\left| #1 \right|} % single pipe
\newcommand{\eps}{\varepsilon}
%\DeclareMathOperator{\fl}{fl}

%\DeclareMathOperator{\acosh}{arccosh} 

% Define a command to write a nice-looking element, e.g. 4,2 He
%\newcommand{\elem}[3]{\ensuremath{{}^{{#1}}_{{#2}}\mathrm{{#3}}}}

% Vector definitions
\newcommand{\mat}[1]{\mathbf{#1}} %matrix is bold upright
\renewcommand{\vec}[1]{\bm{#1}} %vector is bold italic
\newcommand{\op}[1]{\mathsf{#1}} % ``operator'' is sans serif

\newcommand{\vd}{\bm{\cdot}} % slightly bold vector dot
\newcommand{\del}{\vec{\nabla}} % gradient (Del) is bold
\newcommand{\grad}{\vec{\nabla}} % gradient

%\newcommand{\abr}[1]{\langle {#1} \rangle}
\newcommand{\abr}[1]{\left\langle {#1} \right\rangle} % angle brackets for avg.

%% topbox is useful in extended definitions of math terms inside an align
\newcommand{\topbox}[2][0.6]{\parbox[t]{#1\columnwidth}{\raggedright{}#2}}

% commands to make text in math mode appear as zero-width (better-looking
% integrals/sums, e.g.)
% from mathmode.pdf page 74, or Alexander R. Perlis ``A complement to \smash,
% \llap, and \rlap''

\def\mathllap{\mathpalette\mathllapinternal}
	\def\mathllapinternal#1#2{%
	\llap{$\mathsurround=0pt#1{#2}$}%
}
\def\clap#1{\hbox to 0pt{\hss#1\hss}}%
\def\mathclap{\mathpalette\mathclapinternal}%
\def\mathclapinternal#1#2{%
	\clap{$\mathsurround=0pt#1{#2}$}%
}
\def\mathrlap{\mathpalette\mathrlapinternal}%
\def\mathrlapinternal#1#2{%
	\rlap{$\mathsurround=0pt#1{#2}$}%
}

%\setSRJthesisfigurepaths

% For use in the introduction
\newcommand{\chaptersynopsis}[1]{%
\subsubsection{Chapter~\ref{#1}: \nameref{#1}}}
% For referencing equations from a previous chapter; defined like this for
% compatibility with ``single-chapter'' compile
\newcommand{\tagref}[1]{\tag{\ref{#1}}}

% float setup
\renewcommand\floatpagefraction{.70}
\renewcommand\topfraction{.95}
\renewcommand\bottomfraction{.95}
\renewcommand\textfraction{.1}

% Improve list and enum spacing
\renewcommand{\itemhook}{\par\setstretch{1}%
  \setlength{\topsep}{0pt}%
  \setlength{\parskip}{0pt}%
  \setlength{\partopsep}{.5\baselineskip}%
  \setlength{\parsep}{0pt}%
  \setlength{\itemsep}{0pt}%
}
\renewcommand{\enumhook}{\itemhook}

\makeatletter
\renewcommand\section{\@startsection {section}{1}{\z@}%
                                     {-2.5ex  \@minus -.2ex}%
                                     {1.5ex \@minus .2ex}%
                                     {\normalfont\Large\bfseries}}
\renewcommand\subsection{\@startsection{subsection}{2}{\z@}%
                                     {-1.75ex  \@minus -.2ex}%
                                     {1.25ex \@minus .2ex}%
                                     {\normalfont\large\bfseries}}
\renewcommand\subsubsection{\@startsection{subsubsection}{3}{\z@}%
                                     {-1.5ex  \@minus -.2ex}%
                                     {1.25ex \@minus .2ex}%
                                     {\normalfont\normalsize\bfseries}}
\makeatother  
%%%%%%%%%%%%%%%%%%%%%%%%%%%%%%%%%%%%%%%%%%%%%%%%%%%%%%%%%%%%%%%%%%%%%%%%%%%%%%%%
\author{Seth R.~Johnson}
\title{Anisotropic Diffusion Approximations for Time-dependent Particle
Transport\texorpdfstring{\\%
  \emph{DRAFT: \today}}{}
}

\program{Nuclear Engineering and Radiological Sciences}
\degree{Doctor of Philosophy}
\chaircommitteemember{Edward W.~Larsen}{Professor}
\committeemember{Thomas J.~Downar}{Professor}
\committeemember{James P.~Holloway}{Professor}
\committeemember{William R.~Martin}{Professor}
\committeemember{Katsuyo S.~Thornton}{Professor}

%%%%%%%%%%%%%%%%%%%%%%%%%%%%%%%%%%%%%%%%%%%%%%%%%%%%%%%%%%%%%%%%%%%%%%%%%%%%%%%%
\includeonly{%
introduction,%
adDerivation,%
conclusion,%
}
%%%%%%%%%%%%%%%%%%%%%%%%%%%%%%%%%%%%%%%%%%%%%%%%%%%%%%%%%%%%%%%%%%%%%%%%%%%%%%%%
\begin{document}

% let ``align'' break anywhere
\allowdisplaybreaks
% keep huge spaces out from between paragraphs, especially when using double
% spacing
\setlength{\parskip}{0pt plus 0pt minus 0pt}

%%%%%%%%%%%%%%%%%%%%%%%%%%%%%%%%%%%%%%%%%%%%%%%%%%%%%%%%%%%%%%%%%%%%%
% FRONT MATTER
\frontmatter

\maketitle

%%%%%%%%%%%%%%%%%%%%%%%%%%%%%

\begin{frontispiece}
\begin{flushleft}
For in much wisdom is much vexation,\hfill\\
\hspace{1.5em}and he who increases knowledge increases sorrow.
\end{flushleft}
---Eccl.~1:18
%THIS PAGE INTENTIONALLY NOT LEFT BLANK
\end{frontispiece}

%%%%%%%%%%%%%%%%%%%%%%%%%%%%%

\begin{dedication}
  To my parents, Ayn Rand and God.
\end{dedication}

%%%%%%%%%%%%%%%%%%%%%%%%%%%%%

\begin{acknowledgments}
  Acknowledge people here.
\end{acknowledgments}

%%%%%%%%%%%%%%%%%%%%%%%%%%%%%

%\begin{preface}
%  before reading this, you should know\dots
%\end{preface}

% list of contents, etc
\tableofcontents
\listoftables
\listoffigures
%\listofappendices

% % the optional normal abstract is formatted the same as preface and acknowledgements,
% % and is listed in the table of contents
% \begin{abstract}
% \end{abstract}

%%%%%%%%%%%%%%%%%%%%%%%%%%%%%%%%%%%%%%%%%%%%%%%%%%%%%%%%%%%%%%%%%%%%%
% MAIN MATTER
\mainmatter

% note that chapter markers MUST go inside the ``include''-d file
\documentclass[11pt]{SRJresearch}

\let\origmaketitle\maketitle
\renewcommand\maketitle\relax
\renewcommand\contentsname\relax

\author{Seth R.~Johnson}
\title{Anisotropic Diffusion Approximations for Time-dependent Particle
Transport}
\date{\today}

\usepackage{color}
\usepackage{setspace}
\definecolor{intertextbg}{gray}{0.85}

% Modify the chapter output
\makeatletter
\newcommand{\newchapter}{%
\noindent%
\hspace{\stretch{1}}%
\colorbox{intertextbg}{\parbox{0.75\columnwidth}{\centering Chapter from Seth's
dissertation}}%
\hspace{\stretch{1}}
}
\makeatother

% Turn the chapter command into title/figures
\newcommand{\chapter}[1]{%
\title{#1}%
\origmaketitle
\newchapter
{\par\noindent\centering\rule[.25ex]{\columnwidth}{2pt}\par}%
\vspace{-3\baselineskip}
\tableofcontents%
{\par\noindent\centering\rule[.25ex]{\columnwidth}{2pt}\par}%
\clearpage%

\onehalfspacing
}

% set up graphics paths
\setSRJthesisfigurepaths

% Individual documents don't have access to chapter labels
\newcommand{\chaptersynopsis}[1]{\subsection{#1}}
% Create a label instead of referencing from a previous chapter
\newcommand{\tagref}[1]{\label{#1}}

% Kill appendices
%\renewcommand{\appendix}{}

%\newcommand{\index}[1]{}

\usepackage{booktabs} % \toprule, \midrule, \bottomrule
\usepackage{rotating}



\begin{document}

\documentclass[11pt]{SRJresearch}

\let\origmaketitle\maketitle
\renewcommand\maketitle\relax
\renewcommand\contentsname\relax

\author{Seth R.~Johnson}
\title{Anisotropic Diffusion Approximations for Time-dependent Particle
Transport}
\date{\today}

\usepackage{color}
\usepackage{setspace}
\definecolor{intertextbg}{gray}{0.85}

% Modify the chapter output
\makeatletter
\newcommand{\newchapter}{%
\noindent%
\hspace{\stretch{1}}%
\colorbox{intertextbg}{\parbox{0.75\columnwidth}{\centering Chapter from Seth's
dissertation}}%
\hspace{\stretch{1}}
}
\makeatother

% Turn the chapter command into title/figures
\newcommand{\chapter}[1]{%
\title{#1}%
\origmaketitle
\newchapter
{\par\noindent\centering\rule[.25ex]{\columnwidth}{2pt}\par}%
\vspace{-3\baselineskip}
\tableofcontents%
{\par\noindent\centering\rule[.25ex]{\columnwidth}{2pt}\par}%
\clearpage%

\onehalfspacing
}

% set up graphics paths
\setSRJthesisfigurepaths

% Individual documents don't have access to chapter labels
\newcommand{\chaptersynopsis}[1]{\subsection{#1}}
% Create a label instead of referencing from a previous chapter
\newcommand{\tagref}[1]{\label{#1}}

% Kill appendices
%\renewcommand{\appendix}{}

%\newcommand{\index}[1]{}

\usepackage{booktabs} % \toprule, \midrule, \bottomrule
\usepackage{rotating}



\begin{document}

\documentclass[11pt]{SRJresearch}

\input{_preamble.tex}

\begin{document}

\input{../introduction.tex}

\bibliographystyle{ans}
\bibliography{../../SRJall}
\end{document}



\bibliographystyle{ans}
\bibliography{../../SRJall}
\end{document}



\bibliographystyle{ans}
\bibliography{../../SRJall}
\end{document}


\include{trtBackground}
% !TEX root = _individual/adDerivation.tex

%%%%%%%%%%%%%%%%%%%%%%%%%%%%%%%%%%%%%%%%%%%%%%%%%%%%%%%%%%%%%%%%%%%%%%%%%%%%%%%%
\chapter{Anisotropic diffusion}\label{chap:adDerivation}

Chapter text goes here.

An example of a figure using proper, typographically appealing rules with the
\verb|booktabs| package is Table~\ref{tab:angularDomain}.

\begin{table}[htb]
  \centering
  \begin{tabular}{rccc} \toprule
   Geometry & $\vec{\Omega}$ & Domain $S$ & $\ud\Omega$
\\ \midrule
   1-D & $\mu$ & $-1 \le \mu \le 1$ & $\ud\mu$
   \\
   2-D & $\sqrt{1-\mu^2} \cos \omega \vec{i}
   + \sqrt{1-\mu^2} \sin \omega \vec{j}$
   & $-1 \le \mu \le 1$, $0 \le \omega < 2\pi$ & $\ud\mu \ud \omega$
   \\
   Flatland & $\cos \omega \vec{i} + \sin \omega \vec{j}$
   & $0 \le \omega < 2\pi$ & $\ud \omega$
   \\
   3-D & $\mu \vec{i}
   + \sqrt{1-\mu^2} \cos \omega \vec{j}
   + \sqrt{1-\mu^2} \sin \omega \vec{k}$
   & $-1 \le \mu \le 1$, $0 \le \omega < 2\pi$ & $\ud\mu \ud \omega$
\\ \bottomrule
  \end{tabular}
  \caption{Angular variables in the various geometries.}
  \label{tab:angularDomain}
\end{table}


\include{aponeDerivation}
\include{implementation}
\documentclass[11pt]{SRJresearch}

\let\origmaketitle\maketitle
\renewcommand\maketitle\relax
\renewcommand\contentsname\relax

\author{Seth R.~Johnson}
\title{Anisotropic Diffusion Approximations for Time-dependent Particle
Transport}
\date{\today}

\usepackage{color}
\usepackage{setspace}
\definecolor{intertextbg}{gray}{0.85}

% Modify the chapter output
\makeatletter
\newcommand{\newchapter}{%
\noindent%
\hspace{\stretch{1}}%
\colorbox{intertextbg}{\parbox{0.75\columnwidth}{\centering Chapter from Seth's
dissertation}}%
\hspace{\stretch{1}}
}
\makeatother

% Turn the chapter command into title/figures
\newcommand{\chapter}[1]{%
\title{#1}%
\origmaketitle
\newchapter
{\par\noindent\centering\rule[.25ex]{\columnwidth}{2pt}\par}%
\vspace{-3\baselineskip}
\tableofcontents%
{\par\noindent\centering\rule[.25ex]{\columnwidth}{2pt}\par}%
\clearpage%

\onehalfspacing
}

% set up graphics paths
\setSRJthesisfigurepaths

% Individual documents don't have access to chapter labels
\newcommand{\chaptersynopsis}[1]{\subsection{#1}}
% Create a label instead of referencing from a previous chapter
\newcommand{\tagref}[1]{\label{#1}}

% Kill appendices
%\renewcommand{\appendix}{}

%\newcommand{\index}[1]{}

\usepackage{booktabs} % \toprule, \midrule, \bottomrule
\usepackage{rotating}



\begin{document}

\documentclass[11pt]{SRJresearch}

\let\origmaketitle\maketitle
\renewcommand\maketitle\relax
\renewcommand\contentsname\relax

\author{Seth R.~Johnson}
\title{Anisotropic Diffusion Approximations for Time-dependent Particle
Transport}
\date{\today}

\usepackage{color}
\usepackage{setspace}
\definecolor{intertextbg}{gray}{0.85}

% Modify the chapter output
\makeatletter
\newcommand{\newchapter}{%
\noindent%
\hspace{\stretch{1}}%
\colorbox{intertextbg}{\parbox{0.75\columnwidth}{\centering Chapter from Seth's
dissertation}}%
\hspace{\stretch{1}}
}
\makeatother

% Turn the chapter command into title/figures
\newcommand{\chapter}[1]{%
\title{#1}%
\origmaketitle
\newchapter
{\par\noindent\centering\rule[.25ex]{\columnwidth}{2pt}\par}%
\vspace{-3\baselineskip}
\tableofcontents%
{\par\noindent\centering\rule[.25ex]{\columnwidth}{2pt}\par}%
\clearpage%

\onehalfspacing
}

% set up graphics paths
\setSRJthesisfigurepaths

% Individual documents don't have access to chapter labels
\newcommand{\chaptersynopsis}[1]{\subsection{#1}}
% Create a label instead of referencing from a previous chapter
\newcommand{\tagref}[1]{\label{#1}}

% Kill appendices
%\renewcommand{\appendix}{}

%\newcommand{\index}[1]{}

\usepackage{booktabs} % \toprule, \midrule, \bottomrule
\usepackage{rotating}



\begin{document}

\documentclass[11pt]{SRJresearch}

\input{_preamble.tex}

\begin{document}

\input{../flatland.tex}

\bibliographystyle{ans}
\bibliography{../../SRJall}
\end{document}



\bibliographystyle{ans}
\bibliography{../../SRJall}
\end{document}



\bibliographystyle{ans}
\bibliography{../../SRJall}
\end{document}


\include{simpleNumericalResults}
\include{trtNumericalResults}
\documentclass[11pt]{SRJresearch}

\let\origmaketitle\maketitle
\renewcommand\maketitle\relax
\renewcommand\contentsname\relax

\author{Seth R.~Johnson}
\title{Anisotropic Diffusion Approximations for Time-dependent Particle
Transport}
\date{\today}

\usepackage{color}
\usepackage{setspace}
\definecolor{intertextbg}{gray}{0.85}

% Modify the chapter output
\makeatletter
\newcommand{\newchapter}{%
\noindent%
\hspace{\stretch{1}}%
\colorbox{intertextbg}{\parbox{0.75\columnwidth}{\centering Chapter from Seth's
dissertation}}%
\hspace{\stretch{1}}
}
\makeatother

% Turn the chapter command into title/figures
\newcommand{\chapter}[1]{%
\title{#1}%
\origmaketitle
\newchapter
{\par\noindent\centering\rule[.25ex]{\columnwidth}{2pt}\par}%
\vspace{-3\baselineskip}
\tableofcontents%
{\par\noindent\centering\rule[.25ex]{\columnwidth}{2pt}\par}%
\clearpage%

\onehalfspacing
}

% set up graphics paths
\setSRJthesisfigurepaths

% Individual documents don't have access to chapter labels
\newcommand{\chaptersynopsis}[1]{\subsection{#1}}
% Create a label instead of referencing from a previous chapter
\newcommand{\tagref}[1]{\label{#1}}

% Kill appendices
%\renewcommand{\appendix}{}

%\newcommand{\index}[1]{}

\usepackage{booktabs} % \toprule, \midrule, \bottomrule
\usepackage{rotating}



\begin{document}

\documentclass[11pt]{SRJresearch}

\let\origmaketitle\maketitle
\renewcommand\maketitle\relax
\renewcommand\contentsname\relax

\author{Seth R.~Johnson}
\title{Anisotropic Diffusion Approximations for Time-dependent Particle
Transport}
\date{\today}

\usepackage{color}
\usepackage{setspace}
\definecolor{intertextbg}{gray}{0.85}

% Modify the chapter output
\makeatletter
\newcommand{\newchapter}{%
\noindent%
\hspace{\stretch{1}}%
\colorbox{intertextbg}{\parbox{0.75\columnwidth}{\centering Chapter from Seth's
dissertation}}%
\hspace{\stretch{1}}
}
\makeatother

% Turn the chapter command into title/figures
\newcommand{\chapter}[1]{%
\title{#1}%
\origmaketitle
\newchapter
{\par\noindent\centering\rule[.25ex]{\columnwidth}{2pt}\par}%
\vspace{-3\baselineskip}
\tableofcontents%
{\par\noindent\centering\rule[.25ex]{\columnwidth}{2pt}\par}%
\clearpage%

\onehalfspacing
}

% set up graphics paths
\setSRJthesisfigurepaths

% Individual documents don't have access to chapter labels
\newcommand{\chaptersynopsis}[1]{\subsection{#1}}
% Create a label instead of referencing from a previous chapter
\newcommand{\tagref}[1]{\label{#1}}

% Kill appendices
%\renewcommand{\appendix}{}

%\newcommand{\index}[1]{}

\usepackage{booktabs} % \toprule, \midrule, \bottomrule
\usepackage{rotating}



\begin{document}

\documentclass[11pt]{SRJresearch}

\input{_preamble.tex}

\begin{document}

\input{../conclusion.tex}

\bibliographystyle{ans}
\bibliography{../../SRJall}
\end{document}




\bibliographystyle{ans}
\bibliography{../../SRJall}
\end{document}




\bibliographystyle{ans}
\bibliography{../../SRJall}
\end{document}




%\appendix
%\include{pytrt}

%%%%%%%%%%%%%%%%%%%%%%%%%%%%%%%%%%%%%%%%%%%%%%%%%%%%%%%%%%%%%%%%%%%%%%%%%%%%%%%%
% BIBLIOGRAPHY
\backmatter
\bibliographystyle{ans}
\bibliography{../SRJall}
%%%%%%%%%%%%%%%%%%%%%%%%%%%%%%%%%%%%%%%%%%%%%%%%%%%%%%%%%%%%%%%%%%%%%%%%%%%%%%%%
% OPTIONAL INDEX
%\phantomsection %makes sure it points to the right page
%\addtocontents{toc}{chapter}{Index}
%\printindex
%%%%%%%%%%%%%%%%%%%%%%%%%%%%%%%%%%%%%%%%%%%%%%%%%%%%%%%%%%%%%%%%%%%%%%%%%%%%%%%%
\end{document}
